\documentclass[conference]{IEEEtran}

% Judul
\title{Implementasi Algoritma Dijkstra}

% Penulis
\author{\IEEEauthorblockN{Fadiah Mumtaz Andevi}
\IEEEauthorblockA{School of Electrical Engineering and Informatics}
\textit{Institut Teknologi Bandung}\\
Bandung, Indonesia
Email: 18320009@std.stei.itb.ac.id}


\begin{document}


\maketitle

\begin{abstract}
    Kebun Raya Purwodadi dengan luas area sekitar 85 hektar ternyata kekurangan papan informasi yang menyebabkan pengunjung kerap kali kebingungan dalam mencari lokasi tana- man tertentu. Paper ini bertujuan untuk membuat simulasi dari algoritma yang dapat menentukan jarak terdekat antara pengunjung (pengguna program) dengan lokasi tanaman yang dituju. Algoritma yang digunakan adalah algoritma Dijkstra yang beroperasi secara menyeluruh (greedy) untuk menguji seitap persimpangan (Vertex) dan jalan (Edge) pada Kebun Raya Purwodadi. Berdasarkan hasil simulasi dan pengujian, kompleksitas ruang dari program ini adalah O(V) karena adanya pembentukan array yang berisi V nodes untuk mencari heap min- imum. Sementara, kompleksitas waktu dari algoritma tersebut adalah O(V2)

\end{abstract}

\begin{IEEEkeywords}
    Dijkstra, Vertex, Edge, Tanaman.
\end{IEEEkeywords}

\section{Implementasi}
Studi mengenai penggunaan algoritma Dijkstra dalam men- cari jarak terdekat dapat diimplementasikan pada kasus pen- carian tanaman pada Kebun Raya Purwodadi seperti yang telah dilakukan oleh Yusuf et al di tahun 2017 [1]. Paper ini bertu- juan untuk melakukan simulasi kembali terhadap penelitian yang telah dilakukan dengan bahasa C serta mengevaluasi efisiensinya melalui perhitungan kompleksitas waktu dan ru- ang dengan analisis Big-O.

\subsection{graphic}
darisini dapat dilihat grafik.

\section{Kesimpulan}
dapat disimpulkan.


\end{document}

\bibliographystyle{IEEEtran}
\bibliography{reference}